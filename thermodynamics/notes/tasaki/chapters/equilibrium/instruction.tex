以上のように、熱力学系は「外界」からの力学的な操作を通じて理解していくから、
その操作はある程度はっきりさせておく必要がある。
\textbf{
  力学的な操作の間に熱力学的な系が外界に行う仕事は(少なくとも原理的には)
  純粋な力学的な測定から決定できるはずである。
}
\footnote{
  加えた力学的な操作の仕事は測定できそうに思える。
  一方で、熱力学的な系が外界にする仕事が力学的測定によって決定されるはずというのは本当だろうか。
  熱力学的な系の情報のうち、力学的系から制御できる情報として示量変数が考えられるのであって
  熱力学系が外界にする仕事が示量変数として与えられるのだろうか?
}

「何らかの操作により$X_1$から$X_2$に変化させる」というときは、
\begin{quote}
  \begin{itemize}
    \item 力学的な操作
    \item 壁に関する操作
    \footnote{
      壁を差し込む、取り除くという操作に必要な仕事は理想的に0とする。
      熱力学系の一辺の長さが$L$に近い系とすると、系の大きさを変える一般の操作は$L^3$のオーダーであるのに対し、
      壁の挿入、撤去に伴う仕事の大きさは$L^2$のオーダーである。
      この壁の仕事に関する理想化はしばしば使われる。
    }
    \item その組み合わせ
  \end{itemize}
\end{quote}
で$X_1$から$X_2$に変化させることを意味する。