\subsection*{要請2.1 等温環境での平衡状態}
ある環境
\footnote{
  この「環境」もいわば一種の熱力学系であり、同じ役割を果たすような熱力学系を「熱浴」(Heat bath, reservoir)と呼ぶ。
  この熱浴は熱力学の枠組みの中で実現することができるがこの議論は後に譲ることとする。
  いまは、熱浴が注目している系と接触しエネルギーのやり取りを行っていること、
  系に対して十分大きく系に変化が生じたしても、熱浴の温度に変化がないということがわかっていれば十分である。
  例えば、氷水が断熱容器に囲まれているような例を考えると、水または氷がなくなるまでは温度が
  \SI{0}{\celsius}を保つ系として考えられる。
  田崎熱力学においてはこのような相転移を利用した系を「裏技」などと呼んでいる。
}
に熱力学的な系を置き、示量変数の組を固定したまま十分長い時間が経過すると、系が平衡状態(equilibrium state)に達する。
平衡状態では系の性質は時間が経っても変化しない。
同じ環境に置いた系の平衡状態は示量変数の組の値だけで完全に決定される。
\footnote{
  すなわち、ある環境において示量変数の組が同じ2つの系があって平衡状態にあるとき
  その平衡状態は全く同じということ。
}
\footnote{
  一般には、系の内部で温度が一様でなく部分で温度が異なるような平衡状態も考えられる。
  このような状態は複合状態と呼ばれ、一般に本書では扱わない。
  これ以降は、特に明記しない限り単純状態と呼ばれる系の内部で温度が一様な状態を考える。
}

\subsection*{要請2.2 環境と温度}
各々の環境を特徴づける\textbf{温度 Temperature}という実数の量が存在。
環境に置いた熱力学系の平衡状態を左右するのは、環境の温度のみ。
\footnote{
  その中においた熱力学系に対して同じように作用する環境を総称してつけた名前であって、大きな自由度があるはずである。
  この量は我々のこれまでの経験や通常の温度と同じとして考えてもよい。
  このような温度一定の環境が存在すること、環境そのものを仮定としておかないアプローチもある。
  ただ、「熱力学を我々の経験に立脚して理解する」という目的のもとこのような仮定を置く。
}
\footnote{
  経験が示すように\SI{50}{\celsius}の水\SI{100}{\celsius}との水を同体積混合した場合には、\SI{150}{\celsius}の水が得られるわけではないことを知っている。
  温度は系全体を定数倍したとしても、性質が同じであれば変化しない。
  このような量を、示量変数(extensive variables)に対し、示強変数(intensive variables)という。
}

\subsection*{要請2.3 平衡状態の記述}
熱力学系の平衡状態は環境の温度$T$と、示量変数の組$X$、すなわち$(T;X)$という組で完全に決まる。
\footnote{
  ある環境を特徴づけるのは温度$T$であって、その環境内においた平衡状態は示量変数の組$X$で完全に決定される。
}
\footnote{
  厳密には正しくない。三重点とよばれる状態では平衡状態は$T$,$X$を決めても定まらない。
}

\subsection*{要請2.4 断熱系の平衡状態}
熱力学系を断熱壁(Adiabatic wall)で囲い、$X$を固定したまま十分長い時間経過し平衡状態に達したとき、
その状態$(T; X)$ははじめの状態でのみ決まり、周りの環境によらない。
\footnote{
  「熱」の定義をせずとも、系のふるまいを通じてAdiabatic wallを定義している。
}

\subsection*{}
いくつかその他注意すべき点を述べておく。
\begin{quote}
  \begin{itemize}
    \item $(T;X)$が作る数学的空間を\textbf{状態空間(State Space)}という
    \item 状態空間の関数を\textbf{状態量}または\textbf{熱力学関数}という
    \item 
      具体的に流体の系を考えたとき$X=(V, N)$とすれば、$(T;V, N)$という量を決めれば平衡状態はただひとつに決まるとした。
      一方で、分子の大きさよりもわずかに大きいような半径を持つ円筒を考えた場合その中の流体のふるまいは全く異なっているだろう。
      また、その向きによって重力の影響も考えられる。
      このように、$V$を決めたとしても平衡状態が定まらないような場合が考えられる。
      しかし、一般に日常的なスケールの等方的な容器に日常的な密度で物質が入っている場合には
      容器の形状や重力の効果を無視して考える。
      
  \end{itemize}
\end{quote}
