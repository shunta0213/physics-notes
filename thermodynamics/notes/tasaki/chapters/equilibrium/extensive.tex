体積$V'$と物質量$N'$の系と、体積$V''$と物質量$N''$の系を接触させるとその結果としてえられる系は体積$V=V'+V''$物質量$N=N'+N''$のとなる。
このような性質が成り立つとき、$N$, $V$は相加的(additive)という。

一般に、系の全体の大きさを$\lambda$倍したとき、同じように$\lambda$倍される量を\textbf{示量的(extensive)}であるという。

Additiveで、Extensiveな変数$V$、$N$を熱力学系の\textbf{示量変数(extensive variables)}と呼び、$(V, N)$の組を\textbf{示量変数の組}と呼ぶ。
こうすると、
\begin{equation}
  \lambda(V, N) = (\lambda V, \lambda N)
\end{equation}
と書くことができる。

示量変数の組は$(V, N)$ではなく系によって様々である。それを$X$と表現する(と便利である)。
より一般的に$X$という記号を使うが、具体的な系として$X=(V, N)$として考えイメージを持ちながら進めていくとよい。
\footnote{
  $X=(V, N)$に対し$X'$が$(V', N')$を表すこともあれば、$(V', N)$を表すこともある。
}
