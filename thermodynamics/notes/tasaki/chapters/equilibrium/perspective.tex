\textbf{
  示量変数への力学的操作を通じて得られる情報が、われわれが熱力学的な系について知るべきすべてである。
}

熱力学系の外側には、マクロな「力学的な世界」が存在することを前提とする。
この「外界」はNewton力学の体系を想定するのが自然であり、
熱力学的な現象である温度変化、相転移、化学反応などは理解のできない情報である。
「外界」からみた示量変数は熱力学的性質の一部のようにみえる一方で、
熱力学的な示量変数は「外界」の性質ともいえる。
すなわち、これら示量変数を通じてこれらの世界は接続されている。
\footnote{
  具体的に物質量を考えてみよ。熱力学のはじめにはそもそも原子や分子の存在を前提としていなかった。
  一方で「外界」から物質量見たとき、化学反応や相転移といった「謎の現象」によってその値が変化しうる。
}

この示量変数に何らかの変化を起こす際に必要な力学的な仕事を測定することができる。
この「仕事」を「外界」で測定することによって、熱力学的な系の内部の定量的な情報を手に入れようとする試みが熱力学といえる。
\textbf{
  外界から見た熱力学的系は、いわば「動かせる把手のついたブラックボックス」のようなもの。
}
