\documentclass[dvipdfmx,oneside]{jsbook}

\usepackage{amsmath,amsfonts, amssymb}
\usepackage{bm}
\usepackage{siunitx}
\usepackage{graphicx}
\usepackage{booktabs}
\usepackage{multirow}
\usepackage{subcaption}
\usepackage[version=3]{mhchem}
\usepackage{url}
\usepackage{longtable}
\usepackage[unicode,hidelinks,pdfusetitle]{hyperref}
\usepackage{nccmath}
\usepackage{physics}

\newcommand{\TX}{(T,X)}
\newcommand{\TXzero}{(T,X_{0})}
\newcommand{\TXone}{(T, X_1)}
\newcommand{\TXtwo}{(T, X_2)}
\newcommand{\ToneXone}{(T_1, X_1)}
\newcommand{\TtwoXtwo}{(T_1, X_1)}

\newcommand{\iso}{\xrightarrow{i}}
\newcommand{\iq}{\xrightarrow{iq}}
\newcommand{\iqrv}{\overset{iq}{\longleftrightarrow}}
\newcommand{\Wcyc}{W_{cyc}}
\newcommand{\Wmax}{W_{max}}

\newcommand{\FX}{F(T, X)}
\newcommand{\FTVN}{F(T, V, N)}

\newcommand{\pX}{p(T, X)}
\newcommand{\pTVN}{p(T, V, N)}

\newcommand{\muX}{\mu(T, X)}
\newcommand{\muTVN}{\mu(T, V, N)}

\begin{document}

\tableofcontents

\chapter{熱力学とはなにか}

\section{気体の熱力学から普遍的な熱力学へ}

\section{熱力学と普遍性}
\section{本書の内容について}
\begin{quote}
  \begin{itemize}
    \item 熱とはなにか\\熱についての直観多用せず、限定された状況でのみ定義する。ここでの主役は「熱」ではなく操作的に定義する「仕事」である。
    \item Helmholtzの自由エネルギー、内部エネルギーの定義し、Carnotの定理を踏まえたうえで自動的にエントロピーを定義する。
    \item 完全な熱力学関数(Gibbs: fundamental equations)の概念
    \item Legendre変換等の熱力学と直結した数学的な道具の自然的な導入
  \end{itemize}
\end{quote}
\section*{演習}
\paragraph{1.1}
解説参照
\footnote{
  断熱膨張過程では理想気体に温度変化が起きない、断熱準静過程では$T^{\frac{3}{2}}V=$一定とあるが、それはなぜだろうか。
}

\paragraph{1.2}
解説参照
\footnote{
  現実的な系のサイズのこのサイクルの熱効率を考えよ。
}

\paragraph{1.3}
分子、原子の存在をどうして僕らは実在すると信じているのだろうか。
江沢洋「だれが原子をみたか」参照。

\chapter{平衡状態の記述}
メモ
\footnote{
  この章では田崎熱力学の基本的な考え方、原理がまとまっている。
  これらの考え方と原理を認めると、第三章以降美しく熱力学系が記述されていくことが確認できる。
  
  一方で、やや直観的でなく簡単には認めづらいような要請がいくつかあるようにも感じる。
  いくつか複数の文献を読み、ここらは補っていくべきと考える。
}

\section{熱力学的な系の示量変数}
体積$V'$と物質量$N'$の系と、体積$V''$と物質量$N''$の系を接触させるとその結果としてえられる系は体積$V=V'+V''$物質量$N=N'+N''$のとなる。
このような性質が成り立つとき、$N$, $V$は相加的(additive)という。

一般に、系の全体の大きさを$\lambda$倍したとき、同じように$\lambda$倍される量を\textbf{示量的(extensive)}であるという。

Additiveで、Extensiveな変数$V$、$N$を熱力学系の\textbf{示量変数(extensive variables)}と呼び、$(V, N)$の組を\textbf{示量変数の組}と呼ぶ。
こうすると、
\begin{equation}
  \lambda(V, N) = (\lambda V, \lambda N)
\end{equation}
と書くことができる。

示量変数の組は$(V, N)$ではなく系によって様々である。それを$X$と表現する(と便利である)。
より一般的に$X$という記号を使うが、具体的な系として$X=(V, N)$として考えイメージを持ちながら進めていくとよい。
\footnote{
  $X=(V, N)$に対し$X'$が$(V', N')$を表すこともあれば、$(V', N)$を表すこともある。
}

\section{熱力学の視点}
\textbf{
  示量変数への力学的操作を通じて得られる情報が、われわれが熱力学的な系について知るべきすべてである。
}

熱力学系の外側には、マクロな「力学的な世界」が存在することを前提とする。
この「外界」はNewton力学の体系を想定するのが自然であり、
熱力学的な現象である温度変化、相転移、化学反応などは理解のできない情報である。
「外界」からみた示量変数は熱力学的性質の一部のようにみえる一方で、
熱力学的な示量変数は「外界」の性質ともいえる。
すなわち、これら示量変数を通じてこれらの世界は接続されている。
\footnote{
  具体的に物質量を考えてみよ。熱力学のはじめにはそもそも原子や分子の存在を前提としていなかった。
  一方で「外界」から物質量見たとき、化学反応や相転移といった「謎の現象」によってその値が変化しうる。
}

この示量変数に何らかの変化を起こす際に必要な力学的な仕事を測定することができる。
この「仕事」を「外界」で測定することによって、熱力学的な系の内部の定量的な情報を手に入れようとする試みが熱力学といえる。
\textbf{
  外界から見た熱力学的系は、いわば「動かせる把手のついたブラックボックス」のようなもの。
}

\section{操作}
以上のように、熱力学系は「外界」からの力学的な操作を通じて理解していくから、
その操作はある程度はっきりさせておく必要がある。
\textbf{
  力学的な操作の間に熱力学的な系が外界に行う仕事は(少なくとも原理的には)
  純粋な力学的な測定から決定できるはずである。
}
\footnote{
  加えた力学的な操作の仕事は測定できそうに思える。
  一方で、熱力学的な系が外界にする仕事が力学的測定によって決定されるはずというのは本当だろうか。
  熱力学的な系の情報のうち、力学的系から制御できる情報として示量変数が考えられるのであって
  熱力学系が外界にする仕事が示量変数として与えられるのだろうか?
}

「何らかの操作により$X_1$から$X_2$に変化させる」というときは、
\begin{quote}
  \begin{itemize}
    \item 力学的な操作
    \item 壁に関する操作
    \footnote{
      壁を差し込む、取り除くという操作に必要な仕事は理想的に0とする。
      熱力学系の一辺の長さが$L$に近い系とすると、系の大きさを変える一般の操作は$L^3$のオーダーであるのに対し、
      壁の挿入、撤去に伴う仕事の大きさは$L^2$のオーダーである。
      この壁の仕事に関する理想化はしばしば使われる。
    }
    \item その組み合わせ
  \end{itemize}
\end{quote}
で$X_1$から$X_2$に変化させることを意味する。
\section{等温環境での平衡状態}
\subsection*{要請2.1 等温環境での平衡状態}
ある環境
\footnote{
  この「環境」もいわば一種の熱力学系であり、同じ役割を果たすような熱力学系を「熱浴」(Heat bath, reservoir)と呼ぶ。
  この熱浴は熱力学の枠組みの中で実現することができるがこの議論は後に譲ることとする。
  いまは、熱浴が注目している系と接触しエネルギーのやり取りを行っていること、
  系に対して十分大きく系に変化が生じたしても、熱浴の温度に変化がないということがわかっていれば十分である。
  例えば、氷水が断熱容器に囲まれているような例を考えると、水または氷がなくなるまでは温度が
  \SI{0}{\celsius}を保つ系として考えられる。
  田崎熱力学においてはこのような相転移を利用した系を「裏技」などと呼んでいる。
}
に熱力学的な系を置き、示量変数の組を固定したまま十分長い時間が経過すると、系が平衡状態(equilibrium state)に達する。
平衡状態では系の性質は時間が経っても変化しない。
同じ環境に置いた系の平衡状態は示量変数の組の値だけで完全に決定される。
\footnote{
  すなわち、ある環境において示量変数の組が同じ2つの系があって平衡状態にあるとき
  その平衡状態は全く同じということ。
}
\footnote{
  一般には、系の内部で温度が一様でなく部分で温度が異なるような平衡状態も考えられる。
  このような状態は複合状態と呼ばれ、一般に本書では扱わない。
  これ以降は、特に明記しない限り単純状態と呼ばれる系の内部で温度が一様な状態を考える。
}

\subsection*{要請2.2 環境と温度}
各々の環境を特徴づける\textbf{温度 Temperature}という実数の量が存在。
環境に置いた熱力学系の平衡状態を左右するのは、環境の温度のみ。
\footnote{
  その中においた熱力学系に対して同じように作用する環境を総称してつけた名前であって、大きな自由度があるはずである。
  この量は我々のこれまでの経験や通常の温度と同じとして考えてもよい。
  このような温度一定の環境が存在すること、環境そのものを仮定としておかないアプローチもある。
  ただ、「熱力学を我々の経験に立脚して理解する」という目的のもとこのような仮定を置く。
}
\footnote{
  経験が示すように\SI{50}{\celsius}の水\SI{100}{\celsius}との水を同体積混合した場合には、\SI{150}{\celsius}の水が得られるわけではないことを知っている。
  温度は系全体を定数倍したとしても、性質が同じであれば変化しない。
  このような量を、示量変数(extensive variables)に対し、示強変数(intensive variables)という。
}

\subsection*{要請2.3 平衡状態の記述}
熱力学系の平衡状態は環境の温度$T$と、示量変数の組$X$、すなわち$(T;X)$という組で完全に決まる。
\footnote{
  ある環境を特徴づけるのは温度$T$であって、その環境内においた平衡状態は示量変数の組$X$で完全に決定される。
}
\footnote{
  厳密には正しくない。三重点とよばれる状態では平衡状態は$T$,$X$を決めても定まらない。
}

\subsection*{要請2.4 断熱系の平衡状態}
熱力学系を断熱壁(Adiabatic wall)で囲い、$X$を固定したまま十分長い時間経過し平衡状態に達したとき、
その状態$(T; X)$ははじめの状態でのみ決まり、周りの環境によらない。
\footnote{
  「熱」の定義をせずとも、系のふるまいを通じてAdiabatic wallを定義している。
}

\subsection*{}
いくつかその他注意すべき点を述べておく。
\begin{quote}
  \begin{itemize}
    \item $(T;X)$が作る数学的空間を\textbf{状態空間(State Space)}という
    \item 状態空間の関数を\textbf{状態量}または\textbf{熱力学関数}という
    \item 
      具体的に流体の系を考えたとき$X=(V, N)$とすれば、$(T;V, N)$という量を決めれば平衡状態はただひとつに決まるとした。
      一方で、分子の大きさよりもわずかに大きいような半径を持つ円筒を考えた場合その中の流体のふるまいは全く異なっているだろう。
      また、その向きによって重力の影響も考えられる。
      このように、$V$を決めたとしても平衡状態が定まらないような場合が考えられる。
      しかし、一般に日常的なスケールの等方的な容器に日常的な密度で物質が入っている場合には
      容器の形状や重力の効果を無視して考える。
      
  \end{itemize}
\end{quote}


\chapter{等温操作(Isothermal Operation)とHelmholtzの自由エネルギー(Helmholtz free energy)}
\section{Isothermal Operation}
温度一定の環境(ひとつの熱浴)下で、ある平衡状態$\TXone$から別の平衡状態$\TXtwo$へと変化させるこの操作を\bm{等温操作}という。
記号として、$\TXone\iso\TXtwo$と書く。

特に、操作の途中で常に平衡状態にあるとみなせるような極限的な操作を一般に\bm{準静的}といい、準静的な等温操作を等温準静操作といい、$\TXone\iq\TXtwo$と書く。
この操作の全く逆の操作$\TXtwo\iq\TXone$を考えることができる。
まとめて、$\TXone\iqrv\TXtwo$とかく。

\subsection*{要請3.1 Kelvinの原理}
任意の温度で任意の等温サイクルで系が外界にする仕事$\Wcyc$は正ではない。
\begin{align}\label{iso:kelvin}
  \Wcyc \leq 0
\end{align}
第二種永久機関が存在しないことを原理としておいておく。

\subsection*{結果3.2 等温純正サイクルがする仕事}
等温準静サイクルが外界にする仕事は0である。

\paragraph*{証明}
$\TXone\iq\TXtwo\iq\TXone$という過程を考える。Kelvinの原理よりこの過程が外界にする仕事$\Wcyc$は$\Wcyc\leq0$である。
一方で、全く逆の過程を考えたとき、外界にする仕事は$-\Wcyc$でありこれもKelvinの原理より、$-\Wcyc\leq0$である。
以上より、$\Wcyc=0$である。

\subsection*{結果3.3 最大仕事の原理}
最大仕事$\Wmax(\TXone\iso\TXtwo)$は任意の等温準静過程$\TXone\iq\TXtwo$の間に系が外界に行う仕事に等しい。

\paragraph*{証明}
$\TXone\iq\TXtwo\iso\TXone$という過程を考える。Kelvinの原理よりこのサイクルが外界にする仕事$\Wcyc$は$\Wcyc\leq0$である。
また、
\begin{align*}
  \Wcyc&=W(\TXone\iq\TXtwo) + W(\TXtwo\iso\TXone) \\
  &= -W(\TXtwo\iq\TXone) + W(\TXtwo\iso\TXone) \leq 0 \\
  &\therefore W(\TXtwo\iso\TXone) \leq W(\TXtwo\iq\TXone)
\end{align*}
より示された。
但し途中で、$\TXone\iqrv\TXtwo$であることをもちいた。


\section{Helmholtzの自由エネルギー}
ある任意の温度$T$と、示量変数の組の基準$X_0$を決めたとき何らかの操作で到達できる$X$についてHelmholtzの自由エネルギー$F(T, X)$を
\begin{align}\label{helmholtz-def}
  F(T, X) = \Wmax(\TX\iso\TXzero)
\end{align}
と定義する。

$\TXone\iq\TXtwo$という過程を考えたとき、系が外界にする仕事$W$は
\begin{align*}
  W &= \Wmax(\TXone\iq\TXtwo) \\
  &= W(\TXone\iq\TXzero) + W(\TXzero\iq\TXtwo) \\
  &= \Wmax(\TXone\iso\TXzero) - \Wmax(\TXtwo\iso\TXzero) \\
  &= F(T, X_1) - F(T, X_2)
\end{align*}
とかけるわけである。

Helmholtzの自由エネルギーを用いて$\FTVN$と書いたとき、\\
$\displaystyle \pTVN=-\pdv{V}\FTVN$\\
$\displaystyle \muTVN=\pdv{N}\FTVN$\\
と定義する。


\end{document}