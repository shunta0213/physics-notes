\documentclass[dvipdfmx,oneside]{jsbook}

\usepackage{amsmath,amsfonts, amssymb}
\usepackage{bm}
\usepackage{siunitx}
\usepackage{graphicx}
\usepackage{booktabs}
\usepackage{multirow}
\usepackage{subcaption}
\usepackage[version=3]{mhchem}
\usepackage{url}
\usepackage{longtable}
\usepackage[unicode,hidelinks,pdfusetitle]{hyperref}
\usepackage{nccmath}

\begin{document}

\tableofcontents

\chapter{熱力学とはなにか}

\section{気体の熱力学から普遍的な熱力学へ}

\section{熱力学と普遍性}

\section{本書の内容について}
\begin{quote}
  \begin{itemize}
    \item 熱とはなにか\\熱についての直観多用せず、限定された状況でのみ定義する。ここでの主役は「熱」ではなく操作的に定義する「仕事」である。
    \item Helmholtzの自由エネルギー、内部エネルギーの定義し、Carnotの定理を踏まえたうえで自動的にエントロピーを定義する。
    \item 完全な熱力学関数(Gibbs: fundamental equations)の概念
    \item Legendre変換等の熱力学と直結した数学的な道具の自然的な導入
  \end{itemize}
\end{quote}

\section*{演習}
\paragraph{1.1}
解説参照
\footnote{
  断熱膨張過程では理想気体に温度変化が起きない、断熱準静過程では$T^{\frac{3}{2}}V=$一定とあるが、それはなぜだろうか。
}

\paragraph{1.2}
解説参照
\footnote{
  現実的な系のサイズのこのサイクルの熱効率を考えよ。
}

\paragraph{1.3}
分子、原子の存在をどうして僕らは実在すると信じているのだろうか。
江沢洋「だれが原子をみたか」参照。

\chapter{平衡状態の記述}

\section{熱力学的な系の示量変数}
体積$V'$と物質量$N'$の系と、体積$V''$と物質量$N''$の系を接触させるとその結果としてえられる系は体積$V=V'+V''$物質量$N=N'+N''$のとなる。
このような性質が成り立つとき、$N$, $V$は相加的(additive)という。

一般に、系の全体の大きさを$\lambda$倍したとき、同じように$\lambda$倍される量を\textbf{示量的(extensive)}であるという。

Additiveで、Extensiveな変数$V$、$N$を熱力学系の\textbf{示量変数(extensive variables)}と呼び、$(V, N)$の組を\textbf{示量変数の組}と呼ぶ。
こうすると、
\begin{equation}
  \lambda(V, N) = (\lambda V, \lambda N)
\end{equation}
と書くことができる。

示量変数の組は$(V, N)$ではなく系によって様々である。それを$X$と表現する(と便利である)。
より一般的に$X$という記号を使うが、具体的な系として$X=(V, N)$として考えイメージを持ちながら進めていくとよい。
\footnote{
  $X=(V, N)$に対し$X'$が$(V', N')$を表すこともあれば、$(V', N)$を表すこともある。
}


\section{熱力学の視点}
\textbf{
  示量変数への力学的操作を通じて得られる情報が、われわれが熱力学的な系について知るべきすべてである。
}

熱力学系の外側には、マクロな「力学的な世界」が存在することを前提とする。
この「外界」はNewton力学の体系を想定するのが自然であり、
熱力学的な現象である温度変化、相転移、化学反応などは理解のできない情報である。
「外界」からみた示量変数は熱力学的性質の一部のようにみえる一方で、
熱力学的な示量変数は「外界」の性質ともいえる。
すなわち、これら示量変数を通じてこれらの世界は接続されている。
\footnote{
  具体的に物質量を考えてみよ。熱力学のはじめにはそもそも原子や分子の存在を前提としていなかった。
  一方で「外界」から物質量見たとき、化学反応や相転移といった「謎の現象」によってその値が変化しうる。
}

この示量変数に何らかの変化を起こす際に必要な力学的な仕事を測定することができる。
この「仕事」を「外界」で測定することによって、熱力学的な系の内部の定量的な情報を手に入れようとする試みが熱力学といえる。
\textbf{
  外界から見た熱力学的系は、いわば「動かせる把手のついたブラックボックス」のようなもの。
}


\section{操作}
以上のように、熱力学系は「外界」からの力学的な操作を通じて理解していくから、
その操作はある程度はっきりさせておく必要がある。
\textbf{
  力学的な操作の間に熱力学的な系が外界に行う仕事は(少なくとも原理的には)
  純粋な力学的な測定から決定できるはずである。
}
\footnote{
  加えた力学的な操作の仕事は測定できそうに思える。
  一方で、熱力学的な系が外界にする仕事が力学的測定によって決定されるはずというのは本当だろうか。
  熱力学的な系の情報のうち、力学的系から制御できる情報として示量変数が考えられるのであって
  熱力学系が外界にする仕事が示量変数として与えられるのだろうか?
}

「何らかの操作により$X_1$から$X_2$に変化させる」というときは、
\begin{quote}
  \begin{itemize}
    \item 力学的な操作
    \item 壁に関する操作
    \footnote{
      壁を差し込む、取り除くという操作に必要な仕事は理想的に0とする。
      熱力学系の一辺の長さが$L$に近い系とすると、系の大きさを変える一般の操作は$L^3$のオーダーであるのに対し、
      壁の挿入、撤去に伴う仕事の大きさは$L^2$のオーダーである。
      この壁の仕事に関する理想化はしばしば使われる。
    }
    \item その組み合わせ
  \end{itemize}
\end{quote}
で$X_1$から$X_2$に変化させることを意味する。

\end{document}